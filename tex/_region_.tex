\message{ !name(Notes.tex)}\documentclass{article}
\usepackage[tmargin=1in,bmargin=1in,lmargin=1.5in,rmargin=1.5in]{geometry}
\usepackage{amsfonts,amsmath,amssymb,amsthm,relsize,fancyhdr,parskip,graphicx}

\pagestyle{fancy}
\lhead{Ben Carriel}
\chead{Math 6110 Notes}
\rhead{\today}

\parskip 7.2pt
\parindent 8pt

\DeclareMathOperator{\N}{\mathbb{N}}
\DeclareMathOperator{\Z}{\mathbb{Z}}
\DeclareMathOperator{\Q}{\mathbb{Q}}
\DeclareMathOperator{\R}{\mathbb{R}}
\DeclareMathOperator{\C}{\mathbb{C}}
\DeclareMathOperator{\capchi}{\raisebox{2pt}{$\mathlarger{\mathlarger{\chi}}$}}

\DeclareMathOperator{\divides}{\mathrel{|}}
\DeclareMathOperator{\suchthat}{\mathrel{:}}

\DeclareMathOperator{\lra}{\longrightarrow}
\DeclareMathOperator{\into}{\hookrightarrow}
\DeclareMathOperator{\onto}{\twoheadrightarrow}
\DeclareMathOperator{\bijection}{\leftrightarrow}

\newcommand{\problem}[1]{\noindent{\textbf{Problem #1}}\\}
\newcommand{\problempart}[1]{\noindent{\textbf{(#1)}}}

\newcommand{\der}[2]{\frac{\partial #1}{\partial #2}}
\newcommand{\norm}[1]{\|#1\|}
\newcommand{\diam}[1]{\text{diam}(#1)}

\DeclareMathOperator{\im}{\text{im}}

\newtheorem*{thm}{\\ Theorem}
\newtheorem*{lem}{\\ Lemma}
\newtheorem*{claim}{\\ Claim}
\newtheorem*{defn}{\\ Definition}
\newtheorem*{prop}{\\ Proposition}

\begin{document}

\message{ !name(Notes.tex) !offset(625) }
  This is an algebra of sets and so we apply the Carath\'{e}odory
  Extension theorem to extend this to the $\sigma$-algebra genrated by
  the measurable rectangles. Hence, we have that whenever $E \in
  \mathfrak{M}$, the result of the theorem holds for $\capchi_E$.

  To make the final extension, we note that any open set in $\R^d -
  \{0\}$ can be written as a countable union of rectangles $ \bigcup_j
  A_j \times B_j$, where $A_j$ and $B_j$ are open in $(0,\infty)$ and
  $S^{d-1}$ respectively (The proof of this fact is
  clear. Exercise). It follows that any open set is in $\mathfrak{M}$,
  and therefore so is any Borel set. So we have that the theorem is
  true for $\capchi_E$ whenever $E$ is a Borel set in $\R^d -
  \{0\}$. Then we clearly get the result for all characteristic
  functions of Lebesgue measurable sets because they can be decomposed
  into a Borel set and a set of measure zero. We then complete the
  proof by the familiar construction of going from characteristic
  functions, to simple functions, and then using the Monotone
  convergence theorem to get the general result.
\end{proof}


\message{ !name(Notes.tex) !offset(631) }

\end{document}