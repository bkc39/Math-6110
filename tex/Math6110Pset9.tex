\documentclass{article}
\usepackage[tmargin=1in,bmargin=1in,lmargin=1.5in,rmargin=1.5in]{geometry}
\usepackage{amsfonts,amsmath,amssymb,amsthm,relsize,fancyhdr,parskip,graphicx}

\pagestyle{fancy}
\lhead{Ben Carriel}
\chead{Math 6110 Problem Set 9}
\rhead{\today}

\parskip 7.2pt
\parindent 8pt

\DeclareMathOperator{\N}{\mathbb{N}}
\DeclareMathOperator{\Z}{\mathbb{Z}}
\DeclareMathOperator{\Q}{\mathbb{Q}}
\DeclareMathOperator{\R}{\mathbb{R}}
\DeclareMathOperator{\C}{\mathbb{C}}
\DeclareMathOperator{\capchi}{\raisebox{2pt}{$\mathlarger{\mathlarger{\chi}}$}}

\DeclareMathOperator{\divides}{\mathrel{|}}
\DeclareMathOperator{\suchthat}{\mathrel{|}}

\DeclareMathOperator{\lra}{\longrightarrow}
\DeclareMathOperator{\into}{\hookrightarrow}
\DeclareMathOperator{\onto}{\twoheadrightarrow}
\DeclareMathOperator{\bijection}{\leftrightarrow}

\newcommand{\problem}[1]{\noindent{\textbf{Problem #1}}\\}
\newcommand{\problempart}[1]{\noindent{\textbf{(#1)}}}

\newcommand{\der}[2]{\frac{\partial #1}{\partial #2}}
\newcommand{\norm}[1]{\|#1\|}
\newcommand{\diam}[1]{\text{diam}(#1)}

\DeclareMathOperator{\im}{\text{im}}

\newtheorem*{thm}{\\ Theorem}
\newtheorem*{lem}{\\ Lemma}
\newtheorem*{claim}{\\ Claim}
\newtheorem*{defn}{\\ Definition}
\newtheorem*{prop}{\\ Proposition}

\begin{document}
\problem{6.7.2}
\problempart{a} We will first verify that $\overline{\mathcal{M}}$ is indeed a $\sigma$-algebra. It is clear that $\emptyset, X \in \overline{\mathcal{M}}$ because $\mathcal{M} \subset \overline{\mathcal{M}}$. Now we will see that $\overline{\mathcal{M}}$ is closed under countable unions. Let $\{E_n \cup Z_n\}_{n=1}^\infty$ be a collection of sets in $\overline{\mathcal{M}}$ where each of the $E_n \in \mathcal{M}$ and each $Z_n$ is contained in some set $F_n \in \mathcal{M}$ such that $\mu(F_n) = 0$ for all $n$. So,
\[
\bigcup_{n=1}^\infty (E_n \cup Z_n) = \left(\bigcup_{n=1}^\infty E_n\right) \cup \left(\bigcup_{n=1}^\infty Z_n\right) 
\] 
Clearly, the first term on the right is in $\mathcal{M}$, because $\mathcal{M}$ is closed under countable unions. We then observe that because each of the $Z_n \subset F_n$ that we have $\bigcup_n Z_n \subset \bigcup_n F_n$ and that
\[
\mu\left(\bigcup_n F_n\right) \leq \sum_{n=1}^\infty \mu(F_n) = 0
\] 
So that 
\[
\bigcup_{n=1}^\infty (E_n \cup Z_n) \in \overline{\mathcal{M}}
\]
We now need to check closure under complements. Pick some set $E \cup Z \in \overline{\mathcal{M}}$, such that $Z \subset F$ and $F \in \mathcal{M}$. So
\begin{align*}
(E \cup Z)^c &= E^c \cap Z^c \\
&= E^c \cap (F^c \cup (F - Z)) \\
&= (E^c \cap F^c) \cup (E^c \cap (F-Z))
\end{align*}
Clearly, $(E^c \cap F^c) \in \mathcal{M}$. So we are left to verify that $E^c \cap (F - Z) \subset G$ where $G \in \mathcal{M}$ has measure zero. Note that $(E^c \cap F^c) \in \mathcal{M}$ and that $E^c \cap (F-Z) \subset F$ because $(F-Z) \subset F$ and the intersection preserves the superset, $F$. Hence, we take $G = F$ and we are done. 

%TODO finish proof that this is the smallest one.
  
\problempart{b} Now we need to verify that $\bar{\mu}$ is a measure. It is clear that $\bar{\mu}(\emptyset) = \mu(\emptyset) = 0$. Now we observe that
\begin{align*}
\bar{\mu}\left(\bigcup_{n=1}^\infty (E_n \cup Z_n)\right) &= \bar{\mu} \left((\cup_n E_n) \cup (\cup_n Z_n)\right) \\ 
&\leq \bar{\mu} \left((\cup_n E_n) \cup (\cup_n F_n)\right) \\
&= \mu\left((\cup_n E_n) \cup (\cup_n F_n)\right) \\
&= \mu(\cup_n E_n) \\
&= \sum_{n=1}^\infty \mu(E_n) \\
&= \sum_{n=1}^\infty \bar{\mu}(E_n \cup Z_n)
\end{align*}
So $\bar{\mu}$ is a measure. We will now show that $\bar{\mu}$ is complete. Indeed, take $E \cup Z \in \overline{\mathcal{M}}$ such that $\bar{\mu}(E \cup Z) = 0$. For $Z \subset F$ with $\mu(F) = 0$ we have that $E \cup Z \subset E \cup F$. Because we have
\[
\bar{\mu}(E \cup Z) = 0 = \mu(E)
\]
we have that $E$ is a measure zero set in $\mathcal{M}$. So for $E' \in \mathcal{M}$ with $E' \subset E \cup Z$ we have  $E' \subset E\cup F$ and
\[
\bar{\mu}(E') \bar{\mu}(\emptyset \cup E') = \mu(\emptyset) = 0
\]
So $\bar{\mu}$ is complete. 

\problem{6.7.3} Suppose that $E$ is Carath\'{e}odory measurable and $m(E) < \infty$. Fix an $\epsilon > 0$ and find a $G \in G_\delta$ such that $m_*(E) = m_*(G)$. Because $E$ is measurable we can see
\[
m_*(G - E) = m_*(G) - m_*(G \cap E)= m_*(G) - m_*(E) < \epsilon
\]
To establish the infinite case we use the standard process of intersecting $E$ with the balls of radius $n$ about the origin, and measuring those. More precisely, define $E_n = B_n(0) \cap E$, where $B_n(0)$ is the ball of radius $n$ centered at the origin. Because each of the $E_n$ is measurable ( intersection of measurable sets) we can find a set $G_n \in G_\delta$ such that $m_*(G_n - E_n) < 2^{-n}\epsilon$. So if we set $G = \bigcup_{n=1}^\infty G_n$ then we have that $G \supset E$ and
\[
m_*(G - E) = m_*\left(\left(\bigcup_n G_n\right) - E\right) \leq \sum_{n=1}^\infty m_*(G_n - E) \leq \sum_{n=1}^\infty m_*(G_n - E_n) \leq \epsilon
\] 
Which shows that $E$ is Lebesgue measurable (in the sense of Chapter 1).

Now we suppose that $E$ is Lebesgue measure (in the sense of Chapter 1) and deduce that it is Carath\'{e}odory measurable. Indeed, fix an $A \subset \R^n$ and choose an open set $G$ such that $m_*(G - E) < \epsilon$. So 
\[
A \cap E^c = (A \cap G^c) \cup (A \cap (G - E))
\]
And because $G$ is open and therefore measurable we have
\begin{align*}
m_*(A \cap E) + m_*(A \cap E^c) &\leq m_*(A \cap E) + m_*(A \cap G^c) + m_*(A \cap (G - E)) \\
&\leq m_*(A \cap G) + m_*(A \cap G^c) + m_*(G - E) \\
&\leq m_*(A) + \epsilon
\end{align*}
Because $\epsilon$ was arbitrary, we can let it tend to zero yielding
\[
m_*(A) \geq m_*(A \cap E) + m_*(A \cap E^c)
\]
And so $E$ is Carath\'{e}odory measurable.

\problem{6.7.4} Take some subset $E \subset S^{d - 1}$. By definition we have that $\sigma(E) = d \cdot m(\tilde{E})$ where 
\[
\tilde{E} = \{x \in \R^d \suchthat x/\|x\| \in E, 0 < \|x\| < 1\}
\]
Then if $r$ is a rotation of $\R^d$ then $\sigma(rE) = d\cdot m(r\tilde{E})$. So we must have that $x \in r\tilde{E}$ means that $x = \gamma\psi$ for some $\gamma \leq 1$ and $\psi \in rE$. Equivalently we must have that $x = \gamma r(\alpha)$ for that same $\gamma$ and $\alpha \in E$. Because rotations are linear we see $x = r(\gamma\alpha)$, so $x \in r(\tilde{E})$. As a result we have
\[
m(rE) = d\cdot m(r\tilde{E}) = m(E)
\]
Meaning that $r$ preserves measure on the sphere. 

\problem{6.7.10}
\problempart{a} We have that $\nu_1 \perp \mu$ and $\nu_2 \perp \mu$. This means we can find two pairs of disjoint sets $A_1, B_1$ and $A_2, B_2$ such that $\nu_1(E) = \nu_1(A_1 \cap E)$ and $\mu(E) = \mu(A_1 \cap E)$, and likewise for $A_2$ and $B_2$. We then take $A = A_1\cup A_2$ and $B = B_1 \cap B_2$ and propose that $(\nu_1 + \nu_2)(E) = (\nu_1 + \nu_2)(A \cap E)$ and $\mu(E) = \mu(B \cap E)$. Note that this will imply that $(\nu_1 + \nu_2) \perp \mu$ if $A$ and $B$ are disjoint. We begin by verifying that $A \cap B = \emptyset$. This is clear because $A_1 \cap B \subset A_1 \cap B_1 = \emptyset$ and $A_2 \cap B \subset A_2 \cap B_2 = \emptyset$. So for measurable $E$ we have
\[
\mu(E) = \mu(E \cap B_1) = \mu(E\cap B) + \mu(E \cap (B_1 - B_2)) 
\]
And also
\[
\mu(B_1 - B_2) = \mu((B_1 - B_2) \cap B_2) = 0
\]
Which gives that $\mu(E) = \mu(E \cap B)$. We then compute
\[
\nu_1(E) = \nu_1(E \cap A_1) = \nu_1(E \cap A) - \nu_1(E \cap (A - A_1)) 
\]
We use the same trick as able to see that
\[
\nu_1(A - A_1) = \nu_1((A - A_1) \cap A_1) = 0
\] 
So that $\nu_1(E) = \nu_1(E \cap A)$. In the same vein, we see that $\nu_2(E) = \nu_2(E \cap A)$. This means that $(\nu_1 + \nu_2)(E) = (\nu_1 + \nu_2)(E\cap A)$. Hence, $\nu_1 + \nu_2 \perp \mu$

\problempart{b} This one is straightforward. We are given that $\nu_1 << \mu$ and $\nu_2 << \mu$. Hence, if $\mu(E) = 0$ then we must have that $\nu_1(E) = \nu_2(E) = 0$. So clearly $(\nu_1 + \nu_2)(E) = 0$.

\problempart{c} Because $\nu_1 \perp \nu_2$ we can find disjoint $A,B$ such that $\nu_1(E) = \nu_1(E \cap A)$ and $\nu_2(E) = \nu_2(E \cap B)$. Then we see that
\[
|\nu_1|(E) = \sup \sum_{j} |\nu_1(E_j)| = \sup \sum_j |\nu_1(E_j \cap A)| = |\nu_1|(E\cap A)
\]
The same argument holds to get $|\nu_2|(E) = |\nu_2|(E\cap B)$. So $|\nu_1|$ and $|\nu_2|$ are supported on disjoint sets and $|\nu_1|\perp|\nu_2|$

\problempart{d} Suppose that $|\nu|(E) = 0$. Then we must have that $\sup \sum_j |\nu(E_j)| = 0$ and so $\nu(E_j) = 0$ for every $j$. And so $\nu(E) = 0$. 

\problempart{e} We have that $\nu \perp \mu$ and $\nu << \mu$. By the first condition we can find disjoint $A,B$ such that $\nu(E) = \nu(E\cap A)$ and $\mu(E) = \mu(E\cap B)$. So for measurable $E$ we have that $\mu(E\cap A) = \mu((E\cap A) \cap B) = 0$ because $A \cap B = \emptyset$. So $\nu(E) - \nu(E\cap A) = 0$ because $\mu(E\cap A) = 0$ and $\nu << \mu$ yielding that $\nu = 0$.  

\problem{6.7.16}
\problempart{a} The first thing that we verify is that $\mu$ is translation invariant because
\begin{align*}
\mu(R + x) &= \mu((R_1 + x) \times \cdots \times (R_d + x_d)) \\
&= \mu(R+x_1)\cdots\mu(R_d + x_d) \\
&= \mu(R_1)\cdots\mu(R_d) \\
&= \mu(R)
\end{align*}
Whenever $R$ is a measurable rectangle in the above. So the outer measure $\mu_*$ generated by coverings by rectangles is translation invariant. Hence, $\mu_*$ restricted to $\mathcal{M}$ is translation invariant. So $\mu$ is a multiple of the Lebesgue measure . So $\mu(\mathbb{T}^d) = m(Q)$ modulo the correspondence between these two spaces. 

\problempart{b} This part is straightforward. Suppose that $f$ is measurable. Then $f^{-1}(U)$ is measurable whenever $U$ is measurable. This means that $\tilde{f}(U) \R^d$ is measurable and so $\tilde{f}$ is measurable. Note that $\tilde{f}^{-1}(U) = f^{-1}(U) + \Z^d$ is measurable if and only if $f^{-1}(U)$ is because of translation invariance. The proof above is the same for continuous functions except you replace the word ``measurable'' by ``continuous''.  

\problempart{c} Observe that
\begin{align*}
\int_{\mathbb{T}^d} |(f * g)(x)| &= \int_{\mathbb{T}^d}\left|\int_{\mathbb{T}^d} f(x-y)g(y)dy\right|dx \\
&\leq \int_{\mathbb{T}^d}\int_{\mathbb{T}^d}|f(x-y)||g(y)|dydx \\ 
&= \int_{\mathbb{T}^d}|f(x-y)|dx\int_{\mathbb{T}^d}|g(y)|dy \\
\end{align*}
Because both functions are integrable we see that $\int_{\mathbb{T}^d} |(f * g)(x)| < \infty$, so that $f * g < \infty$ almost everywhere. We then set $u = x-y$ to get
\[
(f*g)(x) = \int_{\mathbb{T}^d} f(x-y)g(y)dy = \int_{\mathbb{T}^d} f(u)g(x-u)du = (g*f)(x)
\]
and we are done. 

\problempart{d} We saw above that $(f*g)$ is integrable and we know that $|e^{-2\pi inx}| = 1$ and so the function $(f*g)(x)e^{-2\pi inx}$ is also integrable. We  compute
\begin{align*}
\int_{\mathbb{T}^d} (f*g)(x)e^{-2\pi inx}dx &= \int_{\mathbb{T}^d} e^{-2\pi inx}\left(\int_{\mathbb{T}^d} f(x-y)g(y)dy\right)dx \\
&= \int_{\mathbb{T}^d} g(y) \left(\int_{\mathbb{T}^d} f(x-y)e^{-2\pi inx}dx\right)dy \\
&= \int_{\mathbb{T}^d} g(y) (e^{-2\pi iny})(e^{-2\pi iny})\left(\int_{\mathbb{T}^d} f(x-y)e^{-2\pi inx}dx\right)dy \\
&= \int_{\mathbb{T}^d} g(y)e^{-2\pi iny}\left(\int_{\mathbb{T}^d} f(x-y)e^{-2\pi in(x-y)}dx\right)dy \\
&= \int_{\mathbb{T}^d} a_ng(y)e^{-2\pi iny} \\
&= a_nb_n
\end{align*}
Which is precisely the same as
\[
f*g \sim \sum a_nb_n e^{-2\pi inx}
\]

\problempart{e} Normality is obvious. We need to show that each of these functions are orthogonal. To see this note that
\[
\int_{\mathbb{T}^d} e^{-2\pi inx}e^{-2\pi imx}dx = \int_{\mathbb{T}^d} e^{2\pi i(m-n)x}dx = \prod_{j=1}^d \int_\mathbb{T} e^{2\pi i(m_j - n_j)x_j}dx_j
\]
We then note that 
\[
\int_\mathbb{T} e^{2\pi i(m_j - n_j)x_jdx_j} = \delta_{m_j}^{n_j}
\]
Where $\delta$ is the Kronecker delta function. This gives that 
\[
\int_{\mathbb{T}^d} e^{-2\pi inx}e^{-2\pi imx}dx =  \prod_{j=1}^d \delta_{m_j}^{n_j} = \prod_{j=1}^d \delta_{m}^{n}
\]
We now need to show that the space is complete. We reduce to the case where $d = 1$ by Fubini's theorem because $f$ is integrable and $|e^{-2\pi in_kx_k}| = 1$. So we see that if $(f, e^{2\pi inx}) = 0$ then
\begin{align*}
(f, e^{2\pi inx}) &= \int_{\mathbb{T}^d} f(x)e^{-2\pi in x}dx \\
&= \int_\mathbb{T} e^{-2\pi in_1x_1}\int_\mathbb{T} e^{-2\pi in_2x_2}\cdots\int_\mathbb{T} e^{-2\pi in_dx_d}f(x)dx_ddx_{d-1}\cdots dx_1\\
&= 0
\end{align*}
Then define 
\[
F_1(x_1) = \int_\mathbb{T} e^{-2\pi in_2x_2}\cdots\int_\mathbb{T} e^{-2\pi in_dx_d}f(x)dx_ddx_{d-1}\cdots dx_3
\]
then 
\[
\int_\mathbb{T} F_1(x_1)e^{-2\pi in_1x_1}dx_1 = 0
\]
for every $x_1$ and so $F_1(x_1) = 0$ almost everywhere because $\{e^{2\pi inx}\}$ is complete in the single dimensional case. Now we inductively define
\[
F_2(x_1,x_2) = \int_\mathbb{T} e^{-2\pi in_3x_3}\cdots\int_\mathbb{T} e^{-2\pi in_dx_d}f(x)dx_ddx_{d-1}\cdots dx_3
\] 
Then at any $x_1$ we have that $F_2(x_1,x_2)$ is a function of $x_2$ with the property
\[
\int_\mathbb{T} F_1(x_1,x_2)e^{-2\pi in_2x_2}dx_2 = F_1(x_1) = 0
\]
Where the equalities hold almost everywhere. We continue this process $d$ times to see that $f(x_1,x_2,\ldots,x_d) = 0$ almost everywhere. We then apply the fact that $(f, e_n) = 0$ for all $n$ implies that $f = 0$ iff $\{e_n\}$ is an orthonormal basis to see that the set $\{e^{-2\pi inx}\}$ is an orthonormal basis for $L^2(\mathbb{T}^d)$.

\problempart{f} We begin by defining 
\[
g(x) = \begin{cases}
g_\epsilon(x) = \epsilon^{-d} & 0 < x_j \leq \epsilon, j = 1,\ldots, n \\
0                                            & \text{elsewhere in } Q
\end{cases}
\]
Then we clearly have $\int g(x)dx = \int g_\epsilon(x)dx = 1$. Hence
\begin{align*}
|f(x) - (f*g_\epsilon)(x)| &= \left| f(x)\int g_\epsilon(y)dy - \int f(x-y)g_\epsilon(y)\right|dy \\
&\leq \int |f(x) - f(x-y)||g_\epsilon(y)|dy
\end{align*}
Because $f$ is continuous on a compact set, it is uniformly continuous. So this means that given $\tilde{\epsilon} > 0$ there is a $\delta > 0$ such that $|x - y| < \delta$ implies $|f(x) - f(y)| < \epsilon$. So for $\epsilon < \delta$ we have
\[
|f(x) - (f * g_\epsilon(x))| \leq \int \tilde{\epsilon}|g_\epsilon|dy = \tilde{\epsilon}
\]
Which gives that $(f * g_\epsilon)(x) \to f(x)$ uniformly as $\epsilon \to 0$. Now suppose that $f \sim \sum a_ne^{2\pi inx}$ and $g_\epsilon \sim \sum b_n^\epsilon e^{2\pi inx}$. Then we must have that $\sum |a_n|^2 < \infty$ and $\sum |b_n^\epsilon|^2 < \infty$ because $f,g_\epsilon \in L^2$. We then use Cauchy-Schwartz to get $\sum |a_nb_n| < \infty$. This gives that $f * g_\epsilon \in L^1$ because $\hat{f * g_\epsilon} \in L^1$ and we can use Fourier inversion. So we see that 
\[
(f * g_\epsilon)(x) = \sum_{n \in \Z^d} a_nb_ne^{2\pi inx}
\]
where the equality is almost everywhere. Now we can choose an $\epsilon$ such that $|f - (f * g_\epsilon)| < \tilde{\epsilon}/2$. For this choice of $\epsilon$ we have that $\sum_{|n| > N} a_nb_n^\epsilon \to 0$ for sufficiently large $N$. For this $N$ we look at the truncated series $\sum_{|n| \leq N} |a_nb_n|$ so that $\sum_{|n| > N} |a_nb_n^\epsilon| < \tilde{\epsilon}/2$. So that 
\begin{align*}
\left| f(x) - \sum_{|n| \leq N} a_nb_ne^{2\pi inx} \right| &\leq |f(x) - (f*g_\epsilon)(x)| + \left|(f * g_\epsilon)(x) - \sum_{|n| \leq N} e^{2\pi inx}\right| \\
&= |f(x) - (f*g_\epsilon)(x)| + \left|\sum_{|n| \leq N} a_nb_n^\epsilon e^{2\pi inx}\right| \\ 
&\leq |f(x) - (f*g_\epsilon)(x)| + \sum_{|n| > N} |a_nb_n^\epsilon| \\
&\leq \tilde{\epsilon}/2 + \tilde{\epsilon}/2 \\
&= \tilde{\epsilon}
\end{align*}
Which shows that $f$ can be uniformly approximated by finite linear combinations of the exponential functions in the basis. 

\end{document}
